%!TEX root = ../report.tex

\begin{document}
	\let\cleardoublepage\clearpage
\chapter{Problem Statement} \label{sec:problem_statement}
Our previous work results reveal not all databases reacts similarly for different heterogeneous data from robotic applications. Also, there are no concrete data models has been defined in the context of robotic applications. For example, the black box designed for ROPOD project uses MongoDB to store data from different sources such as Ethercat, Zyre, ZMQ, and ROS topic. The data is being transformed into a simple flatten JSON document to store the values. These documents are stored under a single collection which is created for each ROS topic or other sources. Each record holds only the information of data generated by the sensors or application itself, but these values are not useful without additional details for example, who created the data, if it is a robot then what type of robot-generated this data from which location? Then in what context other systems should interpret this data.

\begin{center}
	\lstset{%
		caption=geometry\_msgs/Pose ROS topic,
		basicstyle=\ttfamily\footnotesize\bfseries,
		frame=tb
	}
\begin{lstlisting}
			double timestamp
			double position/x
			double position/y
			double position/z
			double orientation/x
			double orientation/y
			double orientation/z
			double orientation/w
\end{lstlisting}
\end{center}

For example in the black box, geometry\_msgs/Pose ROS topic will be flattened to a simple JSON document which has the data structure mentioned above.

In the above format, 'position/x' is a key and the value will be attached with it. Now only with position x,y,z and orientation x,y,z,w, another system which consumes this data would not be able to say who generated this data or at which location this data is being generated and if the other system is doing mathematical calculation, then this data is missing its own context such as unit, dimensions, etc.

Periodically, these massive amounts of data are dumped and backed up to a file system or cloud. After every test run in the black box, a report is generated using the FMEA tool which contains the information regarding the test and components involved in it. Also, these reports include the file location where the dump is stored. During fault diagnosis, these dumps will be restored manually to the database and fetch data using the querying tool provided by the black box itself. 

This approach is not scalable and inefficient in terms of multi-robot systems since there will be individual database instances running in each robot. Moreover, this querying tool is incapable of making queries on multiple MongoDB instances at a time.

In terms of supporting various types of databases setup for robots,  there is no systematic approach to store and retrieve data from external sources. Also, a well-defined data model hasn't defined yet that can map robot components (e.g., sensors) to a robot and even with the world model (e.g., locations). In this case, no mediator system has been developed before to connect between robots and different databases.
\end{document}
