%!TEX root = ../report.tex

\begin{document}
	\let\cleardoublepage\clearpage
    \begin{abstract}
        In robotic applications, sensor-generated data are often ignored after they used it for making actions and sometimes dumped into persistent storage. Since these dumps are not adequately modeled and stored, it makes hard for someone who wants to replay the experiments or to find faults in the sensor data.  Even it is more difficult to debug the dump if there are multi-robots involved in a task. It is probable that different vendors/developers develop multi-robots with different database instances and attribute names to store their sensor dumps which introduces heterogeneity in the sensor data. Heterogeneity introduces additional problems like interoperability issues when sharing data between other robots and also with fault diagnosis tools. One way to overcome these issues is by employing a mediator component as a middle man for all robots and even for humans. In our approach, we designed a mediator architecture which solves integrating sensor data from different databases which are running in various robots. Also, the data modeling issue from the current Black box system is addressed with the help of creating an extendable generic data model for each critical entities in the robot system such as Task, Robot, Sensor, and Observations. Lastly, sensor observations interoperability issue is solved by adding meaning contexts to all the entities, and it is achieved by using JSON-LD data representation. Overall mediator component is developed with GraphQL as a base framework and JSON-LD to represent the response data. This choice of GraphQL and JSON-LD provides further advantages to the system such as a single query language to fetch sensor data regardless of databases used in the robots and context-based data model.
    \end{abstract}
\end{document}
