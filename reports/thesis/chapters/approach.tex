%!TEX root = ../report.tex

\begin{document}
	\let\cleardoublepage\clearpage
\chapter{Concept and Methodology}\label{sec:concept_and_methodology}

\section{Approach}\label{sec:approach}
To find the best data models for robotic applications and build a scalable mediator, we begin with SOA analysis \ref{stateof} to find out approaches that have been followed through before for similar data integration applications. After defining the data model, we will collect a list of recent querying techniques and review them based on the features and possibility of adopting them with the mediator as a base.  For review, we would like to consider current well-known querying techniques such as Graphql, and Falcor. At first, our mediator will support only the databases which are selected based on the results from our previous research work \cite{ravichandranworkbench} and other data sources used by ROPOD such as OpenStreetMap. Then, schema's will be defined to map the data being generated by the robot and the data sources. To reduce the complexity of identifying appropriate data-sources by the robot, in our architecture mediator will dynamically choose the data-source respective to the type of data that robots want to store and retrieve. Still, the configuration will be adjustable according to the scenarios. The architecture proposed in this research is a general design and can be used by anyone who wants to store and analyze the data from the multi-robot system contextually. Finally, to show how one can configure the mediator to an external tool to visualize the heterogeneous data from the mediator in a meaningful way, we used GraphiQL \footnote{\href{https://github.com/graphql/graphiql}{https://github.com/graphql/graphiql}} GUI web application to make queries against the mediator system.
\end{document}
